%-------------------------
% Resume in Latex
% Author : Jake Gutierrez
% Based off of: https://github.com/sb2nov/resume
% License : MIT
%------------------------

\documentclass[letterpaper,11pt]{article}

\usepackage{latexsym}
\usepackage[empty]{fullpage}
\usepackage{titlesec}
\usepackage{marvosym}
\usepackage[usenames,dvipsnames]{color}
\usepackage{verbatim}
\usepackage{enumitem}
\usepackage[hidelinks]{hyperref}
\usepackage{fancyhdr}
\usepackage[english]{babel}
\usepackage{tabularx}
\input{glyphtounicode}


%----------FONT OPTIONS----------
% sans-serif
% \usepackage[sfdefault]{FiraSans}
% \usepackage[sfdefault]{roboto}
% \usepackage[sfdefault]{noto-sans}
% \usepackage[default]{sourcesanspro}

% serif
% \usepackage{CormorantGaramond}
% \usepackage{charter}


\pagestyle{fancy}
\fancyhf{} % clear all header and footer fields
\fancyfoot{}
\renewcommand{\headrulewidth}{0pt}
\renewcommand{\footrulewidth}{0pt}

% Adjust margins
\addtolength{\oddsidemargin}{-0.5in}
\addtolength{\evensidemargin}{-0.5in}
\addtolength{\textwidth}{1in}
\addtolength{\topmargin}{-.5in}
\addtolength{\textheight}{1.0in}

\urlstyle{same}

\raggedbottom
\raggedright
\setlength{\tabcolsep}{0in}

% Sections formatting
\titleformat{\section}{
  \vspace{-4pt}\scshape\raggedright\large
}{}{0em}{}[\color{black}\titlerule \vspace{-5pt}]

% Ensure that generate pdf is machine readable/ATS parsable
\pdfgentounicode=1

%-------------------------
% Custom commands
\newcommand{\resumeItem}[1]{
  \item\small{
    {#1 \vspace{-2pt}}
  }
}

\newcommand{\resumeSubheading}[4]{
  \vspace{-2pt}\item
    \begin{tabular*}{0.97\textwidth}[t]{l@{\extracolsep{\fill}}r}
      \textbf{#1} & #2 \\
      \textit{\small#3} & \textit{\small #4} \\
    \end{tabular*}\vspace{-7pt}
}

\newcommand{\resumeSubSubheading}[2]{
    \item
    \begin{tabular*}{0.97\textwidth}{l@{\extracolsep{\fill}}r}
      \textit{\small#1} & \textit{\small #2} \\
    \end{tabular*}\vspace{-7pt}
}

\newcommand{\resumeProjectHeading}[2]{
    \item
    \begin{tabular*}{0.97\textwidth}{l@{\extracolsep{\fill}}r}
      \small#1 & #2 \\
    \end{tabular*}\vspace{-7pt}
}

\newcommand{\resumeSubItem}[1]{\resumeItem{#1}\vspace{-4pt}}

\renewcommand\labelitemii{$\vcenter{\hbox{\tiny$\bullet$}}$}

\newcommand{\resumeSubHeadingListStart}{\begin{itemize}[leftmargin=0.15in, label={}]}
\newcommand{\resumeSubHeadingListEnd}{\end{itemize}}
\newcommand{\resumeItemListStart}{\begin{itemize}}
\newcommand{\resumeItemListEnd}{\end{itemize}\vspace{-5pt}}

%-------------------------------------------
%%%%%%  RESUME STARTS HERE  %%%%%%%%%%%%%%%%%%%%%%%%%%%%


\begin{document}

%----------HEADING----------
% \begin{tabular*}{\textwidth}{l@{\extracolsep{\fill}}r}
%   \textbf{\href{http://sourabhbajaj.com/}{\Large Sourabh Bajaj}} & Email : \href{mailto:sourabh@sourabhbajaj.com}{sourabh@sourabhbajaj.com}\\
%   \href{http://sourabhbajaj.com/}{http://www.sourabhbajaj.com} & Mobile : +1-123-456-7890 \\
% \end{tabular*}
\vspace{1pt}

\begin{center}
    \textbf{\Huge \scshape Elias Josué HAJJAR LLAUQUEN} \\ \vspace{1pt}
    \small 06 12 53 09 04 $|$ \href{mailto:heliasjosue@protonmail.com}{\underline{HEliasJosue@protonmail.com}} $|$ 
    \href{https://github.com/eliasjhl}{\underline{github.com/eliasjhl}} $|$
    \href{https://eliashajjar.fr}{\underline{eliashajjar.fr}}
\end{center}

\vspace{1pt}
%-----------EDUCATION-----------
\section{Éducation}
  \resumeSubHeadingListStart
    \resumeSubheading
      {Lycée Pablo Picasso}{Perpignan, France}
      {Baccalauréat STI2D - Spécialité SIN (systèmes d'information et numérique)}{Sep. 2021 - Juin 2023}
    \resumeSubheading
      {Epitech}{Montpellier, France}
      {Expert en Ingénierie Logiciel et Expert en Technologies de l'information}{Sep. 2023 -- Juil. 2028}
  \resumeSubHeadingListEnd

\vspace{1pt}
%-----------EXPERIENCE-----------
\section{Expérience}
  \resumeSubHeadingListStart

    \resumeSubheading
      {Fondateur et administrateur de serveur Minecraft}{June 2020 -- Déc. 2022}
      {Projet personnel}{}
      \resumeItemListStart
        \resumeItem{Création et maintenance de plusieurs serveurs reliés entre eux (BungeeCord) et hébergés sur des plateformes différentes (OVH, AWS).}
        \resumeItem{Gestion d'une équipe de modérateurs pour garantir un environnement de jeu convivial et sécurisé.}
      \resumeItemListEnd
      
% -----------Multiple Positions Heading-----------
%    \resumeSubSubheading
%     {Software Engineer I}{Oct 2014 - Sep 2016}
%     \resumeItemListStart
%        \resumeItem{Apache Beam}
%          {Apache Beam is a unified model for defining both batch and streaming data-parallel processing pipelines}
%     \resumeItemListEnd
%    \resumeSubHeadingListEnd
%-------------------------------------------

    \resumeSubheading
      {Stage d'observation de 3ème}{Fév. 2020 -- Fév. 2020}
      {Ecobat Energie}{Perpignan, France}
      \resumeItemListStart
        \resumeItem{Communication avec les clients au sujet des énérgies renouvlables}
        \resumeItem{Mise en place de solutions et appreils éco-responables}
    \resumeItemListEnd

  \resumeSubHeadingListEnd

\vspace{1pt}
%-----------PROJECTS-----------
\section{Projets}
    \resumeSubHeadingListStart
      \resumeProjectHeading
          {\textbf{LibC} $|$ \emph{Librarie C, UNIX}}{Oct. 2023 -- Déc. 2023}
          \resumeItemListStart
            \resumeItem{Recréation des fonctions de la LibC dans le cadre des projets Epitech, en équipe et en solo}
            \resumeItem{Gestion d'équipes avec des outils tels que : Linear, GitHub, GitHub Desktop}
            \resumeItem{Visualized GitHub data to show collaboration}
            \resumeItem{Used Celery and Redis for asynchronous tasks}
          \resumeItemListEnd
      \resumeProjectHeading
          {\textbf{Discord Event bot} $|$ \emph{Github, Flask, Python, Html, Css, sqlite, VueJs}}{Déc. 2023 -- Présent}
          \resumeItemListStart
            \resumeItem{Création d'un bot discord événementielle relié via une API développé en Python utilisant Flask}
            \resumeItem{L'API est relié à la base de données pour que la gestion puisse se faire sur le siteweb et avec le bot discord}
            \resumeItem{Implementation de bot discord sur un serveur communautaire de plus de 400 membres}
          \resumeItemListEnd
      \resumeProjectHeading
          {\textbf{Jeux Unity} $|$ \emph{Unity, GitHub, C\#}}{May 2019 -- Déc. 2022}
          \resumeItemListStart
            \resumeItem{Création de plusieurs jeu en C\# utilisant Unity comme moteur graphique.}
            \resumeItem{TikTokGame : Jeu intéractif utilisant une API non officielle permettant de créer de lives ou les viewers peuvent intéragir}
            \resumeItem{Création également de plusieurs jeu platformer 2D}
          \resumeItemListEnd
      \resumeProjectHeading
          {\textbf{Portfolio} $|$ \emph{VueJs, Html, Css, JavaScript, Netlify (Déploiment)}}{Jan. 2024 -- Fév. 2024}
          \resumeItemListStart
            \resumeItem{Création de mon portfolio personnel pour mettre en avant mes projets personnels}
            \resumeItem{Déploiment du portfolio graĉe à Netlify \& mise en place d'un DNS avec une redirection}
          \resumeItemListEnd
    \resumeSubHeadingListEnd


\vspace{1pt}
%
%-----------PROGRAMMING SKILLS-----------
\section{Compétences techniques}
 \begin{itemize}[leftmargin=0.15in, label={}]
    \small{\item{
     \textbf{Langages}{: Python, C, SQL (Postgres), JavaScript, HTML/CSS} \\
     \textbf{Frameworks}{: Node.js, Flask, Vuejs} \\
     \textbf{Outils}{: Git, Docker, VS Code, Visual Studio, PyCharm, IntelliJ, Clion, RustRover, Pterodactyl} \\
    }}
 \end{itemize}


%-------------------------------------------
\end{document}
